\section{Conclusions}

\subsection{Conclusions}

Throughout this thesis we seeked to answer the question on whether the current implementations of ladder logic have been keeping pace with the technologies and training recieved by today's students. We have discovered that ladder logic although adequate before a time where well trained programmers were abundant have proven to be difficult to use and error prone to a typically modern trained software engineer. Typically ladder logic programs require a great deal of sequential logic and that can only be faked in ladder logic. This not only adds to unintuitiveness but also adds a strain on hardware since modern microcontroller hardware is inherently sequential in nature.

In constructing \plccharts we have shown that a graphical programming language is still possible, and that state diagrams serves as an excellent metaphor. Further more we have shown that with a properly constructed framework \plccharts can be an platform indepedent solution in that multiple hardware can utilize the same \emphasize{Intermediate Language} code generated by our tool. 


\subsection{Summary of Contributions}

\subsubsection{Ladder Logic Analysis}
Ladder logic has been around for nearly 30 years and it remained the unchallanged defacto standard language in the industry. As such no one has ever asked the question as to whether this rather old system has kept up with modern times. Unsuprisingly this thesis has shown with analysis of ladder logic that there are several short comings when compared to a system where we take advantage of modern training the operators have recieved. 


\subsubsection{Development of \plccharts Language}
Part of the thesis looked at current different variants for traditional state diagrams. This analysis led to the conclusion that UML2 was best suited to be modifed into a language that was precise enough to allow for definitions of behaviors of a program so that it may be directly translated into code without ambiguity.

A significant contribution of this thesis comes in the form of the development of \plccharts a new visual programming language based on state diagrams. In this thesis we define the syntax, and semantics of the language. The goal achieved here is that we have successfully created a language that is very similar to UML2 with only minor extensions and is able to define precise executable code without the need of a programmer to fill in the gaps.

In development of \plccharts this thesis also demonstrates how correctness may be demonstrated by analysis of structure and demonstrating that execution paths produce the same traces. This work may be used in the future for any work involving construction of an entirely new visual programming language.


\subsubsection{Development of A Prototype Tool and Hardware Enviroment}
In order to justify the effectiveness of the \plccharts language it was necessary to develop a working tool. The tool contains the programmer's IDE, the Compiler, and the Simulator which are described in detail below.

The main IDE described in section \ref{IDE} presented a significant challange to us. The goal was to use \plccharts to descibe the operation of our system but also to require no program code to be written after the tool compiled our diagrams. In addition the design goals was to make our tool simple to use and to perform as if it was a vector drawing program. This thesis achieves this design goal and the results are shown in \ref{IDE} the IDE usage is similar to any drawing program that you are used to and is extendable to incorporate new drawing elements.

The goal of the compiler was to take our raw diagram and to generate executable code without the need of a ``programmer in the loop''. This compiler which works in conjunction to the models used in the IDE is a significant contribution to any one intending on making a language based on state diagrams. In construction of the compiler which is detailed in section \ref{compiler} challanges were overcome as to how to take a visual metaphor and preserve all the structure of the diagram into the final output of the compiled code. We believe this presents a substantial contribution to a practical view of how visual programming languages may be not only possible but practical.

Finally the simulator borrows much from other state diagram based trace systems. It can highlight and animate the state diagram to help the diagram constructor visualize how their states will play out. In addition to aid in understanding how the system will operate the simulator will also list all variables in a watch list so the outcomes of each state can be carefully monitored against their design.

\subsubsection{Development of A Hardware Platform}
Since Programmable Logic Controllers are embedded devices it was also necessary to construct an embedded platform in order to ensure this thesis is not purely academic. We have developed a hardware platform based on the PIC18F452 chip and the MPLAB C18 C compiler tool chain.

The details fo the construction of the hardware platform and framework are listed in sections \ref{sec:hardwareplatform} and \ref{sec:hardwareframework}. It is important to point out at this point that any hardware the implements the hardware framework will be able to utilize PLCEdit and all the other existing tools. We have designed this hardware platform from the ground up to allow useage of any microcontroller that meets the minimum requirements outlined in \ref{sec:hardwareplatform}. The cross platform nature of this toolchain is unique to our tool chain as conventional Programmable Logic Controllers stay on a propreitary tool chain and programs must be reconstructed to work on different chips.


\section{Future Research}
There was not adequate time during the project to go deeper in to the hardware implementation side of Programmable Logic Controllers. In particular the following features are missing in our design.

\begin{itemize}
	\item In line programming: at present programming requires that a the microcontroller be taken offline it would be ideal that the controller stay operational in order to decrease downtime. In addition taking the microcontroller off of the board is not ideal and therefore a system for in line programming would be a much improved design.
	\item Input and output modules: at present I/O are directly connected to the microcontroller's IO channels. This means that I/O is restricted to the chip's design or buffers should be set up between the chip when higher current loads are required. A more idea design is to untilize dedicated I/O boards that will plug into the main unit much like how commercial Programmable Logic Controllers operate. The added benefit would be in the ability to replace modules as well as have specialized hardware on the I/O boards to deal with special cases.
	\item Plug in architecture for PLCEdit: At present adding a new block to PLCEdit requires edits to the source code. Although every effort has been made to make this an easy affair idealy with a plugin architechture blocks could become modular and not require any modifications to the main source at all.
	\item Sub-diagrams: The ability to break one massive diagram up into sub diagrams would be highly beneficial to a more modularized approach to designing programs.
\end{itemize}

In addition while constructing \plccharts and our tool PLCEdit we realized that the methods used here to program microcontrollers are also applicable to full scale applications with the addition of ``sub-diagrams''. It is quite possible to create a version of the IDE that is capable of creating programs on standard desktop computers and may be worth exploring. We believe that this thesis serves as a good starting point for such an endevour.

For desktop interactive programs state explosion might become an issue expecially in user interfaces. Future research might also look into if incorporating ``sub-diagrams'' is actually sufficient for creating a large scale program.