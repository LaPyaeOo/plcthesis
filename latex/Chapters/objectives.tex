
\chapter{Preface}

\section{Structure of this Thesis}

\noindent
Chapter 2: \emphasize{Overview of Existing Technology}. We will begin by introducing the reader to existing PLC\glossary{name={PLC}, description={Programmable Logic Controller}} implementations. We will go into the history behind programmable logic controllers and their usages. We will also give the user an idea of what kinds of modules manufacturers have created in the industry over time.
\\

\noindent
Chapter 3: \emphasize{Introduction To Ladder Logic}. In this chapter we will expose the reader to the existing language (Ladder Logic) that is currently in use in the industry. The reader will be exposed to the syntax and semantics of this language. In addition the user will also obtain some background insight as to how the language came into looking as it does today.
\\

\noindent
Chapter 4: \emphasize{Software}. The software section covers all of the software implementation information of the proposed language Logic Control Chart. We begin by defining the goals in constructing LCC\glossary{name={LCC},description={Logic Control Chart}}. Next the language is then defined for LCC\glossary{name={LCC},description={Logic Control Chart}}. In the final two sections we go into implementation details of the software package as well as show correctness of the diagram to code translation.
\\

\noindent
Chapter 5: \emphasize{Hardware}. In this section we go into detail about the hardware reference platform. We introduce the hardware framework which allows multiple microcontrollers to be utilized. We finish by showing parts of the hardware driver code that must be implemented should the reader wish to implement their own hardware board.
\\

\noindent
Chapter 6: \emphasize{Summary}. This chapter present all the conclusions of the work. In addition we also recommend future directions this thesis can go if continued.
\\

\noindent
Appendix: Contains examples and diagrams in Ladder Logic.