\chapter{Language}
\section{Problems with Ladder Logic}
Ladder logic because it was based on conventional circuits does not inherently lend itself well to representing how microcontrollers actually operate. It also is difficult to use because human beings think of algorithms and procedures in a sequential fashion. In order for microcontrollers to behave in a parallel fashion analogous to how circuits work a lot of careful but extra design was put in so that the hardware behaved like it was parallel provided everything works fast enough. At the same time in order to work in this paradigm human beings were required to take their sequential logic for their algorithm and use work around to either get the circuits to behave sequentially or to translate their algorithm so that it would work well in a parallel fashion while represented by circuits. Not surprisingly this double translation is a lot of work both for the hardware designers and the programmers when the simpler solution is to instead of pretending the hardware is still parallel just to let the programmer execute his instructions in a sequential fashion (because this is what happens anyways). This makes both the hardware and the programs much simpler to construct as the hardware no longer is required to simulate a method of running that isn't what it natively does and the programmer is also not required to translate his program into circuits.

Suppose a programmer wanted to create the following program in ladder logic: After button A is pressed button B is enabled and will directly control a motor. The corresponding program in ladder logic would look like the following.

%TODO: draw diagram for ladder logic of the program

%\section{State Chart Semantics}\label{sec:lang:statesemantics}
%TODO: this section will cover all semantics of traditional state charts and specific semantics relating to how our program will handle it.

\section{Transitioning from Ladder Logic to State Charts}
TODO: in this section we will attempt to look at example programs from different ladder programs in each category (Basic / Logic / Calculations / Simulated seq logic / Parallel logic). Provide translations into state charts but more importantly discuss how to translate them we might explore how to do this in an automated fashion.

\section{Limitations of State Charts vs Ladder Logic}
TODO: Take a look at programs that are inherently parallel or work in the same fashion as traditional relays. These programs are inherently simpler to express in ladder logic. I expect in the end ladder logic works for really easy programs that can be expressed with relays but as soon as more complex sequencing is required it is just terrible. However show how state charts don't improve things if sequencing is not really required.

\section{The Intermediate Language}\label{sec:IL}
TODO: Show how we are using a C based language but we are writing it in such a way so that it operates as an intermediate language and would be remappable to any microcontroller configuration.