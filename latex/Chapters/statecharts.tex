%Introduction to state charts
\section{State Charts}

Our proposed visual programming language involves using directed state charts with guarded edges. Our languages differs in a few concepts from the usual state charts syntax in order to support features of the hardware. To better understand these differences we will first introduce all the syntax and semantics of a standard UML based state chart.

A State chart is generally defined as a diagram describing the behaviour of a discrete system. Such systems can be either finite state machines, or more complex systems abstracted to resemble a finite state machine. 

%http://en.wikipedia.org/wiki/State_diagram (find different source Taylor atomata paper
Mathematically a state chart can be defined as $M = \lbrace Q, \Sigma, Z, \delta, q_0, F \rbrace$

\begin{itemize}
	\item \textbf{Q:} Set of states.
	\item \textbf{$\Sigma$:} Set of input symbols or actions, these are used when checking guard conditions.
	\item \textbf{Z:} Set of output symbols generated by the system.
	\item \textbf{$\delta$:} Set of state transitions with the mapping $\omega: \Sigma \times Q \rightarrow Z$. These are drawn as arrows and are immediately taken if their guard conditions are true.
	\item \textbf{$q_0$:} Starting state, can be defined as an initial state with no incomming transitions.
	\item \textbf{F:} Accepting states for the system, can also sometimes represent the final state. Drawn as a double outline to the state symbol.
\end{itemize}

A state can be thought of as a set of operating conditions that for example in figure %\ref{fig:state_blink_light}
one such state the light is on.

%digram simple state transition for a blinking light {fig:state_blink_light}

We observe that transitions must always be connected on the tail end and the tip to a state. In addition if a transition can be taken it must be taken immediately. Moore style state machines do not contain outputs on transition edges, Mealy machines do contain an extra output per transition.

%diagram for Moore + Mealy state machines

There are several ways a starting state can be defined as seen in figure 
%\ref{fig:state_moore_mealy}
One such way is to draw a edge that has no state connected to its tail. In our system we choose to use the UML symbol where the start state edge has a solid dot connected to the tail as shown in figure 
%\ref{fig:state_blink_light} // or other uml symbol
.


In a standard state chart a lot of implementation details can be hidden away and the diagram can be rather simple. In this example we see that there's no mention of how to turn on the light or turn off the light. Nor do we care if the light is connected to a particular port.

%Source wikipedia
TODO: The tyoe if state charts that we are using is a DIRECTED GRAPH with GUARDED EDGES. We have a starting state but not necessarily an ACCEPTING state.

TODO: Our state chart is very similar to an UML state diagram.

TODO: We might consider adapting our state chart to resemble UML digrams but that bit is under consideration. currently it does look very similar but edges do not have the correct formatting for guarded conditions on UML.

TODO: Minor detail, all of our states have an implicit edge going to END and will be used if no other conditions are true. We might consider showing the difference in the two.

TODO: Programming side, we need to enable saving because it's becoming a pain in the -bleep-.
