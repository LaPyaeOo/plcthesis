%Introduction to state charts
\section{State Charts}

Our proposed visual programming language involves using directed state charts with guarded edges. Our languages differs in concept from the usual states charts but should not take long to understand. In conventional state charts states are states in your system. The system stays in a state as long as no transition can be taken. Each state in addition does not have a workload accociated with it.

%digram simple state transition for a blinking light

In a standard state chart a lot of implementation details can be hidden away and the diagram can be rather simple. In this example we see that there's no mention of how to turn on the light or turn off the light. Nor do we care if the light is connected to a particular port.

%Source wikipedia
TODO: The tyoe if state charts that we are using is a DIRECTED GRAPH with GUARDED EDGES. We have a starting state but not necessarily an ACCEPTING state.

TODO: Our state chart is very similar to an UML state diagram.

TODO: We might consider adapting our state chart to resemble UML digrams but that bit is under consideration. currently it does look very similar but edges do not have the correct formatting for guarded conditions on UML.

TODO: Minor detail, all of our states have an implicit edge going to END and will be used if no other conditions are true. We might consider showing the difference in the two.

TODO: Programming side, we need to enable saving because it's becoming a pain in the -bleep-.
