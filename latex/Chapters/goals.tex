%goals of this project
\section{Goals}

This project aims to improve on current industrial programmable logic controllers by introducing a more natural graphical programming method. In addition we will evaluate how to create a cost effective alternative using off the shelf parts to constuct our own hardware PLC. In the process we aim to produce a final prototype that will have the same basic feature set of modern PLC's. This includes a main controller unit, a input and output unit, and a prototype of a basic IDE that will work in our new visual langugage paradigm. In this project we propose state charts as the method of choice as it is analogous to all current programming languages. It is also important to this project to understand the deficiencies of ladder logic (the current method). In addition we will evaluate the original use of PLC's and if the old methodologies are still applicable to their modern application. This analyis will provide further understanding on how the original programming methods have been outpaced by more recent technology.

Due to the scope of this project we must deliver several components in order to achieve an acceptable proof of concept. The componets and their subgoals are as follows:

We must deliver a main controller unit with an onboard embedded OS this main controller will serve as our CPU sending commands to the input and output units and run our program. The main controller must be able to execute adequately fast in order to compete against commercial PLC's. Since the user is not concerned with the execution speed of each instruction we measure fast by the time it takes for an input to trigger an output and we compare this to the same semantically equivalent program written for an off the shelf PLC. We hope to achieve this speed up with cheaper hardware by allowing the user to specify instructions in a more natural manner and thus reducing the actual instructions required to perform a task. Our hardware will aim to eliminate the concept of a scan\ref{fig:plcexecution} and in doing so we hope to open up new methods of better making use of available hardware.

This project also aims to deliver a reference input and ouput unit that will be modest in driving capacity but will allow for future upgrades where applications demand more heavy duty hardware. The purpose of the input and output units is to establish a base communication protocol so that new modules can be swappable.

At a higher level we need to deliver an simple proof of concept IDE that will allow the user to enter programs using state charts instead of ladder logic. This new IDE should work like a flow chart drawing program allowing the user to add and remove logic blocks at will. We aim to make this interface as intuitive as possible and also provide a fast efficient translation into machine code in the process. Depending on amount of time available it would be nice to incorporate a debugger to help visualize a program's execution however this might be slated for future work. The main goal of the IDE will remain to provide an easy to understand visual enviroment where the user may enter their program and to prevent where possible user entry error.

The final layer that will be added is a software to hardware compiler that will take user generated programs from the IDE and produce actual machine code. This will be achieved in two parts, first the IDE will compile the diagram into intermediate code this intermediate code will be code that will stay hardware neutral and will resemble C. The intermediate code will contain many calls to functions that will be supported on the chip. These functions will be part of our intermediate language translator that will take the calls placed by intermediate language and translate them into hardware sepcific routines.

By delivering an initial proof of concept software and hardware ecosystem this project would allow for further development for modernizing programmable logic controllers.

